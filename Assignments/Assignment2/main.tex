%\documentclass[journal,12pt,twocolumn]{IEEEtran}
%\usepackage{amsmath}
%\providecommand{\pr}[1]{\ensuremath{\Pr\left(#1\right)}}
%\providecommand{\cbrak}[1]{\ensuremath{\left\{#1\right\}}}
%\newcommand*{\permcomb}[4][0mu]{{{}^{#3}\mkern#1#2_{#4}}}
%\newcommand*{\comb}[1][-1mu]{\permcomb[#1]{C}}


\documentclass[journal,12pt,two column]{IEEEtran}
\renewcommand{\labelenumi}{\alph{enumi})}
%\usepackage[utf8]{inputenc}
%\usepackage{graphicx}
%\usepackage{amsmath}
\usepackage{relsize}
%\begin{document}

\usepackage{setspace}
\usepackage{gensymb}
\usepackage{xcolor}
\usepackage{caption}
%\usepackage{subcaption}
%\doublespacing
\singlespacing

%\usepackage{graphicx}
%\usepackage{amssymb}
%\usepackage{relsize}
\usepackage[cmex10]{amsmath}
\usepackage{mathtools}
%\usepackage{amsthm}
%\interdisplaylinepenalty=2500
%\savesymbol{iint}
%\usepackage{txfonts}
%\restoresymbol{TXF}{iint}
%\usepackage{wasysym}
\usepackage{hyperref}
\usepackage{amsthm}
\usepackage{mathrsfs}
\usepackage{txfonts}
\usepackage{stfloats}
\usepackage{cite}
\usepackage{cases}
\usepackage{subfig}
%\usepackage{xtab}
\usepackage{longtable}
\usepackage{multirow}
%\usepackage{algorithm}
%\usepackage{algpseudocode}
%\usepackage{enumerate}
\usepackage{enumitem}
\usepackage{mathtools}
%\usepackage{iithtlc}
%\usepackage[framemethod=tikz]{mdframed}
\usepackage{listings}


%\usepackage{stmaryrd}


%\usepackage{wasysym}
%\newcounter{MYtempeqncnt}
\DeclareMathOperator*{\Res}{Res}
%\renewcommand{\baselinestretch}{2}
\renewcommand\thesection{\arabic{section}}
\renewcommand\thesubsection{\thesection.\arabic{subsection}}
\renewcommand\thesubsubsection{\thesubsection.\arabic{subsubsection}}

\renewcommand\thesectiondis{\arabic{section}}
\renewcommand\thesubsectiondis{\thesectiondis.\arabic{subsection}}
\renewcommand\thesubsubsectiondis{\thesubsectiondis.\arabic{subsubsection}}

%\renewcommand{\labelenumi}{\textbf{\theenumi}}
%\renewcommand{\theenumi}{P.\arabic{enumi}}

% correct bad hyphenation here
\hyphenation{op-tical net-works semi-conduc-tor}

\lstset{
	language=Python,
	frame=single, 
	breaklines=true,
	columns=fullflexible
}



\begin{document}
	\providecommand{\fourier}{\overset{\mathcal{F}}{ \rightleftharpoons}}
	\providecommand{\ztrans}{\overset{\mathcal{Z}}{ \rightleftharpoons}}
	
	%\providecommand{\hilbert}{\overset{\mathcal{H}}{ \rightleftharpoons}}
	\providecommand{\system}{\overset{\mathcal{H}}{ \longleftrightarrow}}
	%\newcommand{\solution}[2]{\textbf{Solution:}{#1}}
	

	\title{\huge{Assignment 2}\\EE3900}
	\author{\Large{I Sai Pradeep}\\AI21BTECH11013}
    %\today
	\maketitle
    
	\begin{abstract}
	This document contains the solution to Oppenheimer problem 2.9
	\end{abstract}
	\noindent \textbf{Question 2.9:}
Consider the difference equation 
\begin{align}
	\label{eq:question}
  y[n]-\frac{5}{6} y[n-1]+\frac{1}{6} y[n-2]=\frac{1}{3} x[n-1]
\end{align}
\begin{enumerate}
  \item What are impulse response, frequency response, and step response for the casual LTI system satisfying this difference equation.
  \item What is the general form of the homogenous solution of the difference equation?
  \item Consider a different system satisfying the difference equation that is neither casual nor LTI, but that has $y[0]=y[1]=1$. FInd the response of this system to $x[n]= \delta[n]$
\end{enumerate}
\textbf{Solution} 
%\solution
\begin{enumerate}
	\item For system which is causual and LTI, we will have right handed signal
	Applying Z-transform to the equation \eqref{eq:question}, we get,
	\begin{align}
		 \mathcal{Z}(y[n])&-\frac{5}{6} \mathcal{Z}(y[z-1])+\frac{1}{6} \mathcal{Z}(y[n-2])\\ &=\frac{1}{3} \mathcal{Z}(x[n-1])\\
%	\end{align}
%\begin{align}
&\mathcal{Z}(x[n-k])=z^{-k}\mathcal{X}(z)
\end{align}		
	Hence,
	\begin{align}
		\mathcal{Y}(z)-\frac{5}{6} z^{-1}& \mathcal{Y}(z)+\frac{1}{6} z^{-2} \mathcal{Y}(z)=\frac{1}{3} z^{-1} \mathcal{X}(z)\\
		\frac{\mathcal{Y}(z)}{\mathcal{X}(z)}&=\frac{2z^{-1}}{z^{-2}-5z^{-1}+6}
		\\
		\label{eq:yx}
		\frac{\mathcal{Y}(z)}{\mathcal{X}(z)}&=\frac{2z}{(1-2z)(1-3z)}\\
		\mathcal{H}(z)=\frac{\mathcal{Y}(z)}{\mathcal{X}(z)}&=\frac{2z}{(1-2z)(1-3z)}
	\end{align}
\begin{align}
		\label{eq:hz1}
		\mathcal{H}(z)&=\frac{2z}{(1-2z)(1-3z)}
\\
\label{eq:hz}
\mathcal{H}(z)&=2\bigg(\frac{1}{1-3z}-\frac{1}{1-2z}\bigg)
	\end{align}

let $h(n)$ be inpulsive response for the given casual and LTI system. Applying inverse z transform for the equation \eqref{eq:hz}, 
we get,
\begin{align}
h(n)&=\mathcal{Z}^{-1}\bigg(2\bigg(\frac{1}{1-3z}-\frac{1}{1-2z}\bigg)\bigg)\\
h(n)&=2\mathcal{Z}^{-1}\bigg(\frac{-1}{3z}\bigg(\frac{1}{1-\frac{1}{3z}}\bigg)+\frac{1}{2z}\bigg(\frac{1}{1-\frac{1}{2z}}\bigg)\\
h(n)&=2\mathcal{Z}^{-1}\bigg(\bigg(\frac{-1}{3z}(1+\frac{1}{3z}+\frac{1}{9z^{2}}+...\bigg)\\&+\frac{1}{2z}\bigg(1+\frac{1}{2z}+\frac{1}{4z^{2}+...}\bigg)\bigg)\\
h(n)&=2\mathcal{Z}^{-1}\bigg(-\frac{1}{3z}+\frac{1}{9z^{2}}+...+\frac{1}{2z}+\frac{1}{4z^{2}}+...\bigg)\\
h(n)&=2\bigg[\frac{1}{2^{n}}-\frac{1}{3^{n}}\bigg]u[n]
\end{align}
Here ROC is $|z|>\frac{1}{2}$.\\
Frequency response is obtained from the equation \eqref{eq:hz1} by putting $z=e^{j\omega}$
\begin{align}	
	\mathcal{H}(e^{j\omega})&=\frac{2e^{j\omega}}{(1-2e^{j\omega})(1-3e^{j\omega})}\\
\mathcal{H}(e^{j\omega})&=\frac{\frac{1}{3}e^{-j\omega}}{1-\frac{5}{6} e^{-j\omega}+\frac{1}{6} e^{-j2\omega}}
\end{align}
Let $s[n]$ be step response of the given casual and LTI system.
Z transform of $u[n]$ is $\frac{1}{1-z}$
For finding step response, we can put $x[n]=u[n]$, hence placing $\mathcal{X}(z)=\frac{1}{1-z}$ in the equation \eqref{eq:yx}, we get,
\begin{align}
	\frac{\mathcal{Y}(z)}{\frac{1}{1-z}}&=\frac{2z^{-1}}{z^{-2}-5z^{-1}+6}\\
	\mathcal{Y}(z)&=\frac{2z^{-1}}{(z^{-2}-5z^{-1}+6)(1-z)}\\
	\mathcal{Y}(z)&=2(\frac{1}{1-3z}-\frac{1}{1-2z})(\frac{1}{1-z})\\
	\mathcal{Y}(z)&=2(\frac{1}{(1-3z)(1-z)}-\frac{1}{(1-2z)(1-z)})\\
	\label{eq:zsn}	
	\mathcal{Y}(z)&=\frac{1}{1-3z}-\frac{1}{1-z}-2(\frac{1}{1-2z}-\frac{1}{1-z})
\end{align}
Applying inverse z transform to the equation \eqref{eq:zsn}, we get,
\begin{align}
s[n]&=\mathcal{Z}^{-1}(\frac{1}{1-3z}+\frac{1}{1-z}-2 \frac{1}{1-2z})\\
s[n]&=\frac{1}{3}^{n} u[n] +u[n] -2 \frac{1}{2}^{n} u[n] \\
s[n]&=\bigg[-2\frac{1}{2}^{n}+\frac{1}{3}^{n}+1\bigg]u[n]
\end{align}
\item As any input signal can be represented as sum of delta functions, we can write $x[n]$ as 
%\begin{equation}
%x[n]=\mathlarger{\mathlarger{\sum}}_{n=0}^{\infty}‎a_k\delta(n-k)
%\end{equation}
where $a_k$ is a number.\\
As z transform of $\delta[n-k]$ is $z^{-k}$, Hence, from \eqref{eq:yx},
\begin{align}
\mathcal{Y}(z)=\bigg(\frac{1}{1-3z}-\frac{1}{1-2z}\bigg) \mathlarger{\mathlarger{\sum}}_{k=0}^{\infty}a_k z^{-k}\\
\label{eq:boptionZ}
\mathcal{Y}(z)=\mathlarger{\mathlarger{\sum}}_{k=0}^{\infty}a_k \frac{z^{-k}}{1-3z}-\mathlarger{\mathlarger{\sum}}_{k=0}^{\infty}a_k \frac{z^{-k}}{1-2z}
\end{align}  
Applying inverse z transform to the equation \eqref{eq:boptionZ}, we get
\begin{align}
y[n]&=\mathcal{Z}^{-1}\bigg(\mathlarger{\mathlarger{\sum}}_{k=0}^{\infty}a_k \frac{z^{-k}}{1-3z}-\mathlarger{\mathlarger{\sum}}_{k=0}^{\infty}a_k \frac{z^{-k}}{1-2z}\bigg)\\
y[n]&=\mathlarger{\mathlarger{\sum}}_{k=0}^{\infty}A_k \frac{1}{3^{n}}-\mathlarger{\mathlarger{\sum}}_{k=0}^{\infty}B_k \frac{1}{2^{n}}
\end{align}  
Where $A_k$ and $B_k$ are constants.
Hence in general, we can say that the homogenous solotuion of the differential equation would be of the form 
\begin{align}
y[n]=A_1 \bigg(\frac{1}{2}\bigg)^n + A_2 \bigg(\frac{1}{3}\bigg)^n
\end{align}
\item For system which is neither causual nor LTI, we will have left handed. Given $x[n]=\delta[n]$ ,applying inverse z transform to equation \eqref{eq:hz1},we get,
\begin{align}
\mathcal{Y}(z)=\frac{2z}{(1-2z)(1-3z)}\\
\label{eq:c}
\mathcal{Y}(z)=\frac{1}{1-3z}-\frac{1}{1-2z}
\end{align}
Considering left handed signal applying inverse z transform to \eqref{eq:c}, 
\begin{align}
	\label{eq:yn3}
	y[n]&=2\frac{1}{3^{n}}u[-n-1]-2\frac{1}{2^{n}}u[-n-1]\\+&A_1 \bigg(\frac{1}{2}\bigg)^n + A_2 \bigg(\frac{1}{3}\bigg)^n
\end{align}
Given that $y[0]=y[1]=1$. From equation \eqref{eq:yn3}, we get $A_1=4$ and $A_2=-3$
Hence, the response of this system to $x[n]=\delta[n]$ is 
\begin{align}
y[n]&=4\bigg(\frac{1}{2}\bigg)^n-3\bigg(\frac{1}{3}\bigg)^{n}-2\bigg(\frac{1}{2}\bigg)^{n}u[-n-1]\\&+2\bigg(\frac{1}{3}\bigg)^{n}u[-n-1]
\end{align}
		\end{enumerate}
\end{document}
